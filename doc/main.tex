% LTeX: language=en-US
\documentclass[12pt,a4paper,parskip=half-]{scrartcl}
\usepackage[english]{babel}\selectlanguage{english}

\usepackage[utf8x]{inputenc}
\usepackage{lmodern}
\usepackage{graphicx}
\usepackage{xcolor}
% \usepackage{xevlna}
% \usepackage{amssymb,amsmath,amsthm}
% \usepackage{esvect}
% \usepackage{tikz}
% \usepackage{pgf-umlcd}
\usepackage{underscore}
\usepackage{framed}
\usepackage{microtype}
\usepackage[autostyle=true]{csquotes}
% \usepackage{showframe}
% \usepackage{minted}
\usepackage{fnpct}
\usepackage[onehalfspacing]{setspace}
\usepackage{caption}
\DeclareCaptionType{listing}[Listing][List of Code Listings]

\PassOptionsToPackage{hyphens}{url}

\usepackage[breaklinks]{hyperref}
\usepackage{cleveref}
\hypersetup{
  pdfborderstyle={/S/U/W 0.5},
  pdftitle={Lang3 interpreter},
  pdfauthor={Pavel Altmann}
}
\urlstyle{same}

% make footnotes rules as wide as the text
\renewcommand{\footnoterule}{%
  \kern -4pt
  \hrule width \textwidth height 0.4pt
  \kern 3.6pt
}

\renewcommand{\arraystretch}{1.5}

% define helper command for typesetting code
\newcommand{\code}[1]{\texttt{#1}}
\def\CC{{C\nolinebreak[4]\hspace{.05em}\raisebox{.5ex}{\tiny\textbf{++}}}}

\setstretch{1.125}

\begin{document}

\begin{titlepage}
  \begin{figure}[h]
    \centering
    \includegraphics[width=0.7\textwidth]{assets/logo_fakultat_im.png}
  \end{figure}

  \vspace{1cm}

  \begin{center}
    \LARGE{Lang3 interpreter}\\
    \vspace{.5cm}
    \large{Compiler Construction 2025/26}
  \end{center}

  \vfill

  \noindent
  OTH Regensburg \hfill Pavel Altmann --- Matnr. 3536630\\
  Faculty of Computer Science and Mathematics \hfill \today
\end{titlepage}

\tableofcontents

\addcontentsline{toc}{section}{Contents}

\newpage

\section{Introduction}
\label{sec:introduction}

Lang3 is a simple interpreter for the programming language
L3\footnote{Originally, it was a work-in-progress name in reference to VM3, that
  it was supposed to be compiled down to. Although I later switched to a custom
VM, I kept the name as it was used in too many places already.} written in \CC{}
26 using Flex and Bison for lexing and parsing. Execution is handled through a
custom virtual machine that executes the abstract syntax tree.

\subsection{Features}
\label{sec:features}

\begin{itemize}
  \item Clean syntax similar to Lua with hints of Python
  \item Optional semicolon separated statements
  \item Strong dynamic typing
  \item First-class functions and currying
  \item Garbage collector memory management
  \item Multiple name assignment
\end{itemize}

\subsection{Project and file structure}
\label{sec:project_structure}

The project is split into a bunch of libraries and an application that connects
them together and provides a user interface. All the libraries are in the form
of \href{https://en.cppreference.com/w/cpp/language/modules.html}{\CC{} 20
modules}, while the application is a regular \CC{} program.

\begin{itemize}
  \item \code{apps}
    \begin{itemize}
      \item \code{lang3} – the application
    \end{itemize}
  \item \code{cmake} – CMake templates
  \item \code{src}
    \begin{itemize}
      \item \code{ast} – AST nodes
      \item \code{cli} – command line arguments parser
      \item \code{external} – declarations of external libraries\footnote{Only
        the GoogleTest testing framework.}
      \item \code{parser} – lexing and parsing
      \item \code{utils} – generic reusable utilities
      \item \code{vm} – virtual machine
    \end{itemize}
\end{itemize}

Both the application and the libraries use a \code{src} subfolder for their
sources, module interfaces and private headers and an \code{include/<name>}
subfolder for their public headers.

\section{Usage}
\label{sec:usage}

\subsection{Prerequisites}
\label{sec:prerequisites}

To build the application, the following tools have to be installed
and available:

\begin{itemize}
  \item \CC{} 26 compliant compiler with \CC{} 20 module
    support\footnote{Compiler used for development was LLVM Clang, version
    21.1.6. Newer versions of GCC and MSVC should probably work as well.}
  \item \href{https://cmake.org/}{CMake} 3.31 or newer
  \item \href{https://ninja-build.org/}{Ninja}\footnote{Due to limitations of
      CMake's \CC{} module support, only the Ninja and Visual Studio
      backends are
    supported.}
  \item \href{https://flex.sourceforge.io/}{Flex}
  \item \href{https://www.gnu.org/software/bison/}{Bison}
\end{itemize}

\subsection{Compilation}
\label{sec:compilation}

To build the release version using CMake, run:

\begin{verbatim}
cmake -S . -B build -G Ninja -DCMAKE_BUILD_TYPE=Release
cmake --build build
\end{verbatim}

This will produce a binary called \code{lang3} in the \code{build/bin}
directory.

\subsection{Running}
\label{sec:running}

The \code{lang3} binary has a simple command line interface that can be used to
run \code{.l3} files:

\begin{verbatim}
  lang3 [options] [--] <file.l3>
\end{verbatim}

If not file was specified or the \code{-} value was used, the application will
read from the standard input. The following options are available:

\begin{itemize}
  \item \code{-t, --timings} – print timings of each stage
  \item \code{-d, --debug} – enable all debug logging
  \item \code{--debug-lexer} – enable lexer debug logging
  \item \code{--debug-parser} – enable parser debug logging
  \item \code{--debug-ast} – log parsed AST to console
  \item \code{--debug-ast-graph <file.dot>} – save parsed AST to a DOT file
  \item \code{--debug-vm} – enable VM debug logging
\end{itemize}

If any of the lexer, parser, or AST debug flags and none of the \code{debug} or
\code{debug-ast} flags are specified, the application will only parse the code
and exit without executing it.

\section{L3 language}
\label{sec:l3_language}

The language uses a simple keyword oriented syntax inspired by Lua and Python
with focus on being independent on whitespace or newlines and
simultaneously not requiring semicolons or other statement delimiters. It
implements 6 data types: \code{bool}, \code{int}, \code{float}, \code{string},
\code{vector} and \code{nil}. It also supports working with functions as
first-class values–assigning them to variables, passing them as arguments, and
creating new anonymous functions (closures).

\subsection{Syntax}
\label{sec:syntax}

For detailed overview of the syntax, see
\cref{listing:syntax},
\href{run:./examples/demo.l3}{the demo file} of the examples programs. In short,
the language is structured into blocks, statements and expressions. Blocks
contain a list of statements, e.g. a function declaration, variable assignment,
for loop or an if/else. Inside each statement are either expressions--literals,
function calls, unary/binary operators, etc.---or inner blocks. To keep the
grammar unambiguous, expressions can't be written as a top-level statement,
except for function calls.

Whitespace is only used to separate keywords and identifiers from each other;
indentation are newlines are completely ignored. Semicolons are not required but
optionally supported to separate statements.

\begin{listing}
  \begin{framed}
    \small
\begin{verbatim}
# Variable declaration
let a = 1 + 2 + 2 * -3
let b = a * -3

# Function declaration
fn add(a, b)
  return a + b
end
let c = add(a, b)

# Auto-currying
let curried = add(a)
let d = curried(b)

# Control flow and chained comparison
if (0 <= c < 10) and a == b == 0 then
  println("condition is true")
else
  println("condition is false")
end

# Loops
for i in 0..=10 step 2 do println(i) end

# Optimized for functional coding style
let arr = [1, 2, 3, 4]

fn stringify(arr)
  # Multiple assignment
  let x, xs = head(arr)

  if xs then
    return str(x) + ", " + stringify(xs)
  else
    return str(x)
  end
end

println(stringify(arr))

# Functional helpers
println(sum(map(int, ["1", "2", "3", "4"])))
\end{verbatim}
  \end{framed}
  \caption{Short example of L3 code}
  \label{listing:syntax}
\end{listing}

\subsection{Types}
\label{sec:semantics}

The language follows a standard primitive--object model. The primitive types are
\code{bool}, \code{int}, \code{string}, and \code{nil} and are passed by value.
On the other hand, the objects---functions (first-class values), closures, and
vectors---are passed by reference. The string type is a slightly more complex as
it is internally a reference type, but it is immutable and copy-on-write. So
from the user's perspective, it behaves as a primitive.

There are no implicit type conversions between the types, except coercion to
boolean in conditions---zero, \code{nil} and empty collections are considered
falsey and everything else is truthy. In all other cases, a \code{TypeError} is
raised in case of type mismatch.

\subsection{Collections}
\label{sec:collections}

Both vector and a string are iterable collections, which means that they can be
used in for loops, with built-in functions like \code{sum} and \code{map}, or
for the multiple assignment. Both also support indexing (\code{arr[0]}) and
various forms of slicing (\code{head(string)}, \code{take(string, 2)},
\code{slice(string, 0, 2)}, etc.).

In case of vectors, the elements can be of any type and do not need to be the
same type as the vector itself. This makes it, so the vector can be used as a
tuple, e.g. for multiple return values from functions.

\subsection{Mutability}
\label{sec:mutability}

Each variable has an associated mutability, define during declaration. The
default is immutable, but can be set to mutable with the \code{mut} keyword.
This is possible for every declaration, be it single declaration, multiple
declaration or the loop control variable.

\subsection{Currying}
\label{sec:currying}

When function is called with fewer arguments than its required number, the
arguments are stored in a closure and a new function is returned that accepts
the missing arguments. This is called currying or partial application. In the
example above, the \code{add} function is curried and the \code{curried}
variable points to the curried function with the first argument "prefilled". It
serves the purpose of simplifying composition of functions, while also being a
substitute of the unimplemented default arguments.

\subsection{Built-in functions}
\label{sec:builtins}

There are 24 built-in functions listed in the \cref{table:builtins}.

\begin{table}
  \centering
  \begin{tabular}{|l|l|}
    \hline
    \textbf{Function} & \textbf{Description} \\
    \hline
    \code{print(...)} & Print value(s) to standard output without newline \\
    \hline
    \code{println(...)} & Print value(s) to standard output with newline \\
    \hline
    \code{__trigger_gc()} & Trigger garbage collection (debugging utility) \\
    \hline
    \code{assert(cond, [msg])} & Assert that a condition is true,
    throws error if false \\
    \hline
    \code{error(msg)} & Throw an error with a given message \\
    \hline
    \code{input([msg])} & Read a line of input from standard input \\
    \hline
    \code{int(value, [base])} & Convert a value to integer type \\
    \hline
    \code{str(value)} & Convert a value to string type \\
    \hline
    \code{head(vec)} & Return a pair of the first element and the rest \\
    \hline
    \code{tail(vec)} & Return a pair of the rest and the last element \\
    \hline
    \code{len(vec)} & Returns the length of a collection \\
    \hline
    \code{drop(vec, n)} & Removes the first n elements from a list \\
    \hline
    \code{take(vec, n)} & Takes the first n elements from a list \\
    \hline
    \code{slice(vec, start, end)} & Extracts a portion of a list
    between given indices \\
    \hline
    \code{random([min], [max])} & Generate a random integer from the range \\
    \hline
    \code{sleep(millis)} & Pause execution for specified amount of
    milliseconds \\
    \hline
    \code{map(func, coll)} & Apply a function to each element of a
    collection \\
    \hline
    \code{filter(func, coll)} & Filter elements of a collection
    based on a predicate \\
    \hline
    \code{sum(coll)} & Calculate the sum of elements in a collection \\
    \hline
    \code{all(coll)} & Return true if all elements are truthy
    (short-circuiting) \\
    \hline
    \code{any(coll)} & Return true if any element is truthy
    (short-circuiting) \\
    \hline
    \code{count(func, coll)} & Count how many elements satisfy a condition \\
    \hline
    \code{id(value)} & Identity -- return the argument unchanged \\
    \hline
    \code{range(start, end, [step])} & Generate a vector of numbers
    within a given range \\
    \hline
  \end{tabular}
  \caption{Built-in functions}
  \label{table:builtins}
\end{table}

\section{Virtual machine}
\label{sec:virtual_machine}

The virtual machine accepts the AST and recursively executes it. It
keeps track
of the execution flow, variable scopes and the garbage-collector-driven memory
management.

\subsection{Execution flow}

There are 4 main execution modes: \code{Normal}, \code{Return}, \code{Break},
and \code{Continue}. The \code{Normal} mode is the default execution mode and
while it is active, the VM just executes the code. After encountering a return
statement, the VM transitions to the \code{Return} mode. In this mode, the VM
walks back up the AST until it encounters a function call. Then, it
transitions
back to the \code{Normal} mode and return back to the call site. \code{Break}
and \code{Continue} modes work analogously for loops.

There are checks in place, to make sure that the execution flow is consistent
and, e.g. break outside a loop is detected, and an error is raised.

\subsection{Scopes}
\label{sec:scopes}

The VM keeps track of the current scope and the scope stack. Each
block has it's
own associated scope, which is pushed onto the scope stack when the block is
entered. When the block is exited, the scope is popped from the stack.
Furthermore, many statements create their own extra scopes, e.g. the control
variable in a for loop.

When function is defined, a snapshot of the current scope is saved, and later
restored when the function is called. While doing so, the current scope is
pushed on the unused scope stack stack (that is, a stack of unused scope
stacks; what a mouthful).

In each scope, variables are stored as a vector of identifiers and values.
Originally, a map has been used, but since the scopes are usually very small,
it turned out to be far more efficient to use a vector.

\subsection{Memory management}
\label{sec:memory_management}

Memory is managed by a standard mark-and-sweep garbage collector
(GC). Inside a
\code{GCStorage} class, \code{GCValue} objects are stored in a single linked
list, each of which holding an actual \code{Value}s. The rest of
the VM operates
on \code{RefValue}s, which are trivially-copyable wrappers around pointers to
the \code{GCValue}s. When a value is created, it is emplaced in the GC storage
and a new \code{RefValue} is returned.

To keep track of the used and unused values, the VM pushes all newly created
values on a stack. Periodically, the VM runs the mark-and-sweep algorithm to
free unused values. During the mark phase, it walks the stack, the
scope stack,
and the unused scope stack and marks all the values it contains (it writes to
the \code{GCValue}s through the saved \code{RefValue}s). Then, in the sweep
phase, the \code{GCStorage} removes all unmarked values from the list and
simultaneously unmarks all marked values. This means, that only
values that are
still in the current stack frame or reachable from any of the
current scopes are
kept.

\subsection{Exceptions}
\label{sec:exceptions}

Exceptions are handled through the native \CC{} exception mechanism. To
facilitate this and maintain memory-safety, the VM and surrounding classes make
heavy use of the RAII (resource acquisition is initialization) pattern with
preparation in the constructor and cleanup code in the destructor.

Originally, the flow modes \code{Return}, \code{Break}, and \code{Continue} were
also implemented as exceptions, but I was forced to refactor them later as the
exception throwing and catching turned out to be a bottleneck and accounted for
up to 40\% of the runtime in the game-of-life example. I had plans to convert
the L3 exceptions in the same way, but it turned out to be too much work for too
little benefit (they always end the execution anyway, so there are no
performance considerations).

\end{document}
